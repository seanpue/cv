\documentclass[letterpaper,12pt]{article}
% generated by Docutils <http://docutils.sourceforge.net/>
\usepackage{fixltx2e} % LaTeX patches, \textsubscript
\usepackage{cmap} % fix search and cut-and-paste in Acrobat
\usepackage{ifthen}
\usepackage[T1]{fontenc}
\usepackage[utf8]{inputenc}
\setcounter{secnumdepth}{0}
\newcommand{\DUfooter}{
\#\#\#Page\#\#\#
}

%%% Custom LaTeX preamble
% PDF Standard Fonts
\usepackage{mathptmx} % Times
\usepackage[scaled=.90]{helvet}
\usepackage{courier}

%%% User specified packages and stylesheets
% Stylesheet used to format C.V.

%set font size to 12 point (TODO: how?)


\usepackage[letterpaper,margin=1in]{geometry}

%set font to times

\usepackage{times}


% do not justify
\raggedright


\usepackage{titlesec}

% make section titles uppercase+bold
\titleformat{\section}
  {\normalfont\bfseries\uppercase}{\thesection}{1em}{}

% make subsections superscript+bold

\titleformat{\subsection}
  {\normalfont\bfseries\scshape}{\thesection}{1em}{}

% add horizonal line below section titles

\newcommand{\tmpsection}[1]{}
\let\tmpsection=\section
\renewcommand{\section}[1]{\tmpsection{#1}\hrule\vspace{2mm}}

% adjust spacing for titles


\titlespacing\section{0pt}{12pt plus 4pt minus 2pt}{0pt plus 2pt minus 2pt}
\titlespacing\subsection{0pt}{12pt plus 4pt minus 2pt}{6pt plus 2pt minus 2pt}
\titlespacing\subsubsection{0pt}{12pt plus 4pt minus 2pt}{0pt plus 2pt minus 2pt}



% make a new title (get rid of huge whitespace of article
% TODO: @title is showing up as title so have hardcoded
\renewcommand{\maketitle}{
\begin{center}
{\large\bfseries \textit{Curriculum Vitae of A. Sean Pue}}
\end{center}
}

% make all paragraphs hanging by 2.5em (to match line blocks)
\leftskip 2.5em
\parindent -2.5em



%%% Fallback definitions for Docutils-specific commands

% providelength (provide a length variable and set default, if it is new)
\providecommand*{\DUprovidelength}[2]{
  \ifthenelse{\isundefined{#1}}{\newlength{#1}\setlength{#1}{#2}}{}
}

% admonition (specially marked topic)
\providecommand{\DUadmonition}[2][class-arg]{%
  % try \DUadmonition#1{#2}:
  \ifcsname DUadmonition#1\endcsname%
    \csname DUadmonition#1\endcsname{#2}%
  \else
    \begin{center}
      \fbox{\parbox{0.9\textwidth}{#2}}
    \end{center}
  \fi
}

% lineblock environment
\DUprovidelength{\DUlineblockindent}{2.5em}
\ifthenelse{\isundefined{\DUlineblock}}{
  \newenvironment{DUlineblock}[1]{%
    \list{}{\setlength{\partopsep}{\parskip}
            \addtolength{\partopsep}{\baselineskip}
            \setlength{\topsep}{0pt}
            \setlength{\itemsep}{0.15\baselineskip}
            \setlength{\parsep}{0pt}
            \setlength{\leftmargin}{#1}}
    \raggedright
  }
  {\endlist}
}{}

% title for topics, admonitions, unsupported section levels, and sidebar
\providecommand*{\DUtitle}[2][class-arg]{%
  % call \DUtitle#1{#2} if it exists:
  \ifcsname DUtitle#1\endcsname%
    \csname DUtitle#1\endcsname{#2}%
  \else
    \smallskip\noindent\textbf{#2}\smallskip%
  \fi
}

% hyperlinks:
\ifthenelse{\isundefined{\hypersetup}}{
  \usepackage[colorlinks=true,linkcolor=blue,urlcolor=blue]{hyperref}
  \urlstyle{same} % normal text font (alternatives: tt, rm, sf)
}{}
\hypersetup{
  pdftitle={Curriculum Vitae of A. Sean Pue},
}

%%% Title Data
\title{\phantomsection%
  Curriculum Vitae of A. Sean Pue%
  \label{curriculum-vitae-of-a-sean-pue}}
\author{}
\date{}

%%% Body
\begin{document}
\maketitle


\section{Contact Information%
  \label{contact-information}%
}

\begin{DUlineblock}{0em}
\item[] \textbf{e-mail:}
\href{mailto:a@seanpue.com}{a@seanpue.com}
\textbf{twitter:}
@seanpue
\textbf{github:}
seanpue
\item[] \textbf{address:}
B-360 Wells Hall, 619 Red Cedar Road, East Lansing, MI 48824 U.S.A.
\end{DUlineblock}


\section{Academic Appointment%
  \label{academic-appointment}%
}

\begin{DUlineblock}{0em}
\item[] \textbf{2014–}
Associate Professor, Michigan State University
\item[]
\begin{DUlineblock}{\DUlineblockindent}
\item[] Department of Linguistics and Germanic, Slavic, Asian, and African Languages
\item[] Core Faculty, Global Studies in Arts and Humanities
\end{DUlineblock}
\item[] \textbf{2008–14}
Assistant Professor, Michigan State University
\end{DUlineblock}


\section{Research Appointment%
  \label{research-appointment}%
}

\begin{DUlineblock}{0em}
\item[] \textbf{2009–}
Associate Research Scholar in Hindi and Urdu Literature and Studies, Columbia University
\item[]
\begin{DUlineblock}{\DUlineblockindent}
\item[] Department of Middle Eastern, South Asian, and African Studies
\end{DUlineblock}
\item[] \textbf{2006–8}
Research Associate, University of Chicago
\item[]
\begin{DUlineblock}{\DUlineblockindent}
\item[] South Asia Language Resource Center
\end{DUlineblock}
\end{DUlineblock}


\section{Education%
  \label{education}%
}

\begin{DUlineblock}{0em}
\item[] \textbf{2007}
Ph.D., Columbia University
\item[]
\begin{DUlineblock}{\DUlineblockindent}
\item[] Middle East and Asian Languages and Cultures and Comparative Literature and Society
\item[] Title: “The Desert of Continuity: N. M. Rashed, Modernism, and Urdu Poetry”
\end{DUlineblock}
\item[] \textbf{2002}
M. Phil., Columbia University
\item[]
\begin{DUlineblock}{\DUlineblockindent}
\item[] Middle East and Asian Languages and Cultures and Comparative Literature and Society
\end{DUlineblock}
\item[] \textbf{2001}
M.A., Columbia University
\item[]
\begin{DUlineblock}{\DUlineblockindent}
\item[] Middle East and Asian Languages and Cultures
\end{DUlineblock}
\item[] \textbf{1997}
B.A., University of California
\item[]
\begin{DUlineblock}{\DUlineblockindent}
\item[] Religious Studies, South and South East Asian Studies
\item[] Highest Honors, Phi Beta Kappa
\end{DUlineblock}
\end{DUlineblock}


\section{Publications%
  \label{publications}%
}


\subsection{Book%
  \label{book}%
}

\emph{I Too Have Some Dreams: N. M. Rashed and Modernism in Urdu Poetry}.
Berkeley, CA: University of California Press, 2014.


\subsection{Articles%
  \label{articles}%
}

\textbf{2013}
“Rethinking Modernism and Progressivism in Urdu Poetry.”
\emph{Pakistaniaat} 5.1 (2013): 1-15.

\textbf{2012}
“Ephemeral Asia: N. M. Rashed’s A Stranger in Iran.”
\emph{Comparative Literature} 64.1 (2012): 73-92.

\textbf{2011}
“In the Mirror of Ghalib: Postcolonial Reflections on Indo-Muslim Selfhood.”
\emph{Indian Economic and Social History Review} 48.4 (October-December 2011), 571-592.

\textbf{2011}
“N. M. Rashed and Faiz Ahmed Faiz: A Comparative Analysis.”
\emph{Bunyaad: Journal of Urdu Studies}  2 (2011), 3-18.

\textbf{2010}
“Time is God: Temporality in Pakistani Modernism.”
\emph{Journal of Commonwealth and Postcolonial Studies} 16.1 (2009), 102-116.

\textbf{2008}
“Digital Encoding of South Asian Languages: A Contemporary Guide to Unicode and Fonts.”
\emph{South Asia Language Pedagogy and Technology} 1 (2008).

\textbf{2008}
“Web-Browser Extensions for South Asia Language Classrooms.”
\emph{South Asia Language Pedagogy and Technology} 1 (2008).


\subsection{Book Chapters%
  \label{book-chapters}%
}

\textbf{2013}
“Faiz Ahmed Faiz and N. M. Rashed: A Comparative Analysis.”
In \emph{Daybreak: Writings on Faiz},
ed. Yasmeen Hameed.
Karachi: Oxford University Press, 2013.

\textbf{2010}
“Shibli Nomani.”
In \emph{Nationalism in the Vernacular:
Hindi, Urdu and the Literature of Indian Freedom},
edited by Shobna Nijhawan,
171–177.
New Delhi: Permanent Black, 2010.

\textbf{2008} “Krishan Chandar.”
\emph{The Oxford India Anthology of Modern Urdu Literature},
edited by Mehr Farooqi, 56–65.
Delhi: Oxford University Press, 2008.

\textbf{2006}
“Poems of Desire.”
In \emph{Men of the Global South},
edited by Adam Jones,
6–13.
London: Zed Books, 2006.


\subsection{Book Reviews%
  \label{book-reviews}%
}

\textbf{2009}
Review of \emph{A History of Urdu Literature},
by T. Graham Bailey.
\emph{The Book Review} (February 2009).

\textbf{2008}
Review of \emph{Enlightenment in the Colony:
The Jewish Question and the Crisis of Postcolonial Culture},
by Aamir R. Mufti.
\emph{Annual of Urdu Studies} 23 (2008): 274-9.

\textbf{2005}
Review of \emph{Urdu Texts and Contexts},
by C. M. Naim.
\emph{Annual of Urdu Studies} 20 (2005): 288-290.

\textbf{2000}
Review of \emph{Hali’s Musaddas},
translated by Christopher Shackle and Javed Majeed.
\emph{Annual of Urdu Studies} 15 (2000): 612-615.


\subsection{Translations%
  \label{translations}%
}

\textbf{2005} Krishan Chandar, “Irani Pilau.”
\emph{Annual of Urdu Studies} 20 (2005): 203-210.


\section{Presentations%
  \label{presentations}%
}


\subsection{Conference Papers%
  \label{conference-papers}%
}

\textbf{2014}
“Poets in a Muslim Land:
Sufism, Modernity, and Indo-Muslim Artistic Subjectivity.”
Muslim Studies Conference on “Journeys of Practice,”
Michigan State University.

\textbf{2013}
“Translating Rhythm:
Data-Rich Literary Analysis for Understanding the Politics of Literary Form.”
XVI International Conference of the Forum on Contemporary Theory on
“Translation, Comparatism and the Global South,” University of Mysore.

\textbf{2013}
“Free Verse in Urdu: Identity, Influence, and Innovation.”
Annual Conference on South Asia, University of Wisconsin, Madison.

\textbf{2013}
“Bioinformatic Approaches to the Computation of Poetic Meter,”
with Tracy K. Teal and C. Titus Brown.
Shared Horizons: Data, Biomedicine, and Digital Humanities, University of Maryland.

\textbf{2013}
“Modernism and Realism in Late Colonial India.”
American Comparative Literature Association Annual Meeting, Toronto.

\textbf{2010}
“Desert Wandering: The Modern Landscape of Urdu Poetry.”
American Comparative Literature Association Annual Meeting, New Orleans.

\textbf{2010}
“‘Soviet Pantheism’: Modernism and the Critique of Ideology.”
Association for Asian Studies Annual Meeting, Philadelphia.

\textbf{2009}
“Where is Hasan the Potter Now? A Literary Representation of Failed Artistic Personhood.”
Annual Conference on South Asia, University of Wisconsin, Madison.

\textbf{2009}
“Modernism and Colonial Difference.”
American Comparative Literature Association Annual Meeting, Harvard University.

\textbf{2008}
“From a Place of Solitude to a Place of Community: The Desert in Modern Urdu Poetry.”
Association for Asian Studies Annual Meeting, Atlanta.

\textbf{2007}
“In the Mirror of Ghalib.”
Association for Asian Studies Annual Meeting, Boston.

\textbf{2006}
“Parallel to the Horizon: Desire and Duration in Pakistani Modernism.”
Horizons: Comparative Literature Graduate Student Conference, Stanford University.

\textbf{2006}
“Distance at Death: N. M. Rashed and the Progressives.”
Annual Conference on South Asia, University of Wisconsin, Madison.

\textbf{2005}
“\emph{Ham Eshiyai}: Solidarities After Empire.”
Imagining Empire: South Asia Graduate Student Conference, University of Chicago.

\textbf{2005}
“Partition and National Identity: Urdu Debates on Pakistan’s ‘Fundamental Problem.’”
Modern Language Association Annual Convention, Washington, D. C.

\textbf{2005}
“Sheba in Ruins: Urdu Modernism’s Imaginative Geography.”
American Institute of Pakistan Studies Biennial Conference, University of Pennsylvania.

\textbf{2005}
“Modernists and Marxists: A False Opposition?”
Siting South Asia: A Graduate Student Conference, Columbia University.

\textbf{2005}
“Alternative Geographies: Urdu Translations of Modern Persian Poetry.”
American Comparative Literature Association Annual Meeting, Penn State University.


\subsection{Invited Papers%
  \label{invited-papers}%
}

\textbf{2013}
“The Politics of Literary Form: The Poetic Meters of Miraji.”
Contemporary Hindi/Urdu Literature and Arts, Princeton University.

\textbf{2013}
“A Punjabi Critique of Sufi Idiom: N. M. Rashed and Urdu Literary Tradition.”
South Asia Seminar, University of Texas at Austin.

\textbf{2013}
Research Presentation.
Audio Cultures of India: Sound, Science, and History,
Neubauer Collegium for Culture and Society, University of Chicago.

\textbf{2013}
“Temporality and Islam in Urdu Literary Modernism.”
Language and Literatures of the Muslim World,
Muslim Studies Program,
Michigan State University

\textbf{2013}
“Issues in the Digital Humanities for Hindi/Urdu.”
Bharatiya Bhasha Karyakram (Indian Language Programme),
Center for the Study of Developing Societies, New Delhi.

\textbf{2012}
“Ghazals on the Go: Teaching the Culture of Urdu Poetry.”
Center for Language Teaching Advancement,
Professional Development Event,
Michigan State University.

\textbf{2012}
“Mobile-Ready Hindi-Urdu Digital Literature Reader.”
South Asian Language Pedagogy Conference, Yale University.

\textbf{2011}
“The Mobile Frontier of South Asian Language Pedagogy.”
Looking Through the Languages:
South Asian Language Study for the Liberal Arts Conference,
Yale University.

\textbf{2011}
“Hindi, Urdu, and Beyond:
Web-Based Video and Handwriting Widgets for Mobile and Traditional Devices.”
Explorations in Instructional Technology, Michigan State University.

\textbf{2011}
“Faiz the Poet.”
Guftugu: Faiz Ahmed Faiz, A Centennial Celebration,
Center for South Asia Studies, University of California, Berkeley.

\textbf{2011}
“N. M. Rashed and Faiz Ahmed Faiz: A Comparative Analysis.”
Faiz Ahmad Faiz Birth Centenary Colloquium,
Lahore University of Management Science.

\textbf{2011}
“In the Mirror of Ghalib: Postcolonial Reflections on Indo-Muslim Selfhood.”
Lahore University of Management Science.

\textbf{2011}
Response to \emph{The Language of the Gods in the World of Men:
Sanskrit, Culture, and Power in Premodern India} by Sheldon Pollock.
Cosmopolitan and Vernacular Languages: A Global Conversation, University of Michigan.

\textbf{2010}
“Dialogue and Truth: An Introduction to Gandhi and His Global Legacy.”
Kapur Endowment Lecture, Michigan State University.

\textbf{2010}
“Bridging the Language and Literature Divide:
Textual Encapsulation for South Asian Language Pedagogy and Digital Humanities.”
Teaching South Asia: Language Instruction and Literary Culture, Yale University.

\textbf{2010}
“Ephemeral Asia:
N. M. Rashed’s A Stranger in Iran and the Problem of Modernism in Urdu.”
Global Studies Forum, Michigan State University.

\textbf{2009}
“Blending Content for South Asian Language Pedagogy,”
with Manan Ahmed.
Two-day Workshop, South Asia Summer Language Institute,
University of Wisconsin, 2009.

\textbf{2009}
“Temporality and Difference in Pakistani Modernism.”
South Asia Seminar, University of Chicago.

\textbf{2008}
“Temporality in Pakistani Modernism.”
UrduFest: Contemporary and Historical Facets of Urdu and Its Literature,
University of Virginia.

\textbf{2006}
“The Problem of the Vulgar.”
Between Popular Culture and State Ideology:
Urdu literature and Urdu Media in Present-day Pakistan,
Internationales Wissenschaftsforum, Heidelberg.

\textbf{2003}
“The Buried City: N. M. Rashed and Modern Urdu Poetry.”
Sarai @ Center for the Study of Developing Societies, New Delhi.


\subsection{Panels Organized%
  \label{panels-organized}%
}

\textbf{2013}
“Repositioned Realism.”
American Comparative Literature Association Annual Meeting, Toronto

\textbf{2010}
“Landscapes of Cultural Production.”
American Comparative Literature Association Annual Meeting, New Orleans.

\textbf{2010}
“National Culture and Belonging in Pakistan.”
Association for Asian Studies Annual Meeting, Philadelphia.

\textbf{2008}
“The Geography of Urdu: Canon, Metaphor, Community.”
Association for Asian Studies Annual Meeting, Atlanta.

\textbf{2007}
“The Modern Ghalib.”
Association for Asian Studies Annual Meeting, Boston.


\section{Grants%
  \label{grants}%
}

\textbf{2010}
“Hindi-Urdu Blended Teaching Resources,”
South Asian Language Resource Center Pedagogical Resources Grant (\$25,000)

\textbf{2006}
“Digital Urdu Ghazal Reader,”
South Asian Language Resource Center Pedagogical Resources Grant (\$16,800)

\textbf{2004}
“Mir in Cyberspace,”
Center for Advanced Research in Language Acquisition Mini-Grant (\$3000)


\section{Fellowships%
  \label{fellowships}%
}

\textbf{2011}
American Institute of Pakistan Studies
Short Term Research and Lecturing Fellowship
(“N. M. Rashed and Modernism in Urdu Poetry” in Lahore and Islamabad)

\textbf{2005-6}
FLAS Fellowship (Urdu), Columbia University

\textbf{2003}
Fulbright-Hays Doctoral Dissertation Research Abroad Fellowship (India)

\textbf{2003}
American Institute of Pakistan Studies Dissertation Research Grant (Unactivated)

\textbf{1998-2005}
Faculty Fellowship, Columbia University

\textbf{2001}
Columbia University Graduate School of Arts and Sciences Summer Fellowship (London, UK)

\textbf{2000}
FLAS Summer Fellowship (Punjabi in Chandigarh, India), Columbia University

\textbf{1997-8}
Berkeley Urdu Language Program in Pakistan Fellowship (Lahore)


\section{Research Awards%
  \label{research-awards}%
}

\textbf{2012-3}
MSU College of Arts and Letters Faculty Learning Community Grant,
“Digital Humanities,” with Danielle DeVoss

\textbf{2012}
MSU College of Arts and Letters Research Award,
“South Asian Digital Literary Services”

\textbf{2012}
MSU External Connections Grant (with Frances Pritchett, Columbia University)

\textbf{2012}
MSU Center for Language Teaching Advancement Research Grant,
“Ghazals on the Go: Teaching the Culture of Urdu Poetry”

\textbf{2010}
MSU College of Arts and Letters “Think Tank” Curriculum Development Grant,
“Global Publics”

\textbf{2009} MSU Global Studies in the Arts and the Arts and Humanities Research Grant

\textbf{2009} MSU Blended Teaching Community Research Grant


\section{Prizes%
  \label{prizes}%
}

\textbf{2013}
Global Outlook::Digital Humanities Essay Competition,
First Prize,
for “Bioinformatic Approaches to the Computational Analysis of Urdu Poetic Meter,”
with Tracy K. Teal and C. Titus Brown.


\section{Teaching%
  \label{teaching}%
}


\subsection{Michigan State University%
  \label{michigan-state-university}%
}

AL 340: Digital Humanities Seminar (Spring 2013, Spring 2014)

GSAH 311: Partition, Displacement, and Cultural Memory (Fall 2013)

GSAH 220: Global Espionage: Identity, Intelligence, Power (Fall 2012)

GSAH 230: Encountering Difference: East-West, North South (Fall 2009, Fall 2010, Fall 2011)

IAH 211B: Gandhi’s India in History, Literature, and Film (Spring 2009, Spring 2010)

LL151.2: Basic Hindi I (Fall 2008, Fall 2009, Fall 2010, Fall 2012)

LL152.2 Basic Hindi II (Spring 2009, Spring 2010, Spring 2013)

LL251.2: Intermediate Hindi I (Fall 2010, Fall 2011, Fall 2013))

LL252.2: Intermediate Hindi II (Spring 2014)


\subsection{University of Chicago%
  \label{university-of-chicago}%
}

Third- and Fourth-Year Hindi-3: Modern Hindi Poetry (Spring 2008)

Third- and Fourth-Year Urdu-1: Urdu Short Story (Autumn 2007)


\section{Service%
  \label{service}%
}


\subsection{National%
  \label{national}%
}

\textbf{2013–16}
American Institute of Pakistan Studies, Executive Committee

\textbf{2009-}
American Institute of Pakistan Studies, Board of Trustees


\subsection{Michigan State University%
  \label{id2}%
}

\textbf{2013-}
Asian Studies Center Advisory Board

\textbf{2012-}
Global Studies in Arts and Humanities Advisory Committee

\textbf{2009-10}
Global Studies in Arts and Humanities Planning Committee

\textbf{2009-}
Core Faculty Member, Muslim Studies

\textbf{2008-12}
Global Studies in Arts and Humanities Curriculum Committee

\textbf{2008-}
Core Faculty Member, Asian Studies Center

\textbf{2008-}
Consulting Faculty Member, Gender in Global Context Center

\textbf{2008-}
Integrated Media/Digital Humanities Committee, College of Arts and Letters

\textbf{2008-}
Contractual Core Faculty Member, Global Studies in Arts and Humanities


\subsection{Journal Editorial Boards%
  \label{journal-editorial-boards}%
}

\textbf{2011–}
\emph{Bunyaad: Journal of Urdu Studies}

\textbf{2013–}
\emph{Sagar: A South Asian Research Journal}


\section{Languages%
  \label{languages}%
}

\textbf{research:} Hindi, Urdu, Persian

\textbf{secondary research:} Bengali, Punjabi, Arabic

\textbf{reading:} French, German

\textbf{programming:} Python, Perl, Mathlab, R, Java, Javascript, PHP, XSLT

\DUadmonition[admonition-the-current-pdf-version-of-this-c-v-is-available-at-http-seanpue-com-cv-pdf]{
\DUtitle[admonition-the-current-pdf-version-of-this-c-v-is-available-at-http-seanpue-com-cv-pdf]{The current PDF version of this C.V. is available at \url{http://seanpue.com/cv.pdf}}

This C.V. was last updated on 20 August 2014.
}

\end{document}
